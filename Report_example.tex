\documentclass[14pt,Otchet]{diplomwork}
\reporttype{по учебной практике}

\usepackage{fancyvrb}

\date{2020}
\author{студент МКб-4301-51-00}{Фамилых Имярек Батькович}
\advisor{канд. физ.-матем. наук, доцент}{Иванов Петр Сидорович}

\napravlenie{02.03.01}{Математика и компьютерные науки}
\profile{Математические основы компьютерных наук}
\kafedra{фундаментальной математики}{Е.\,М.~Вечтомов}
\department{компьютерных и физико-математических наук}{Н.\,А.~Бушмелева}
\institute{математики и информационных систем}



\usepackage{longtable,array, hhline}% тонкая настройка таблиц

\renewcommand{\theFancyVerbLine}{\footnotesize\arabic{FancyVerbLine}}


\newcommand{\alert}[1]{{\color{red}#1}} 
\sloppy

\begin{document}

\maketitle
\newpage

\tableofcontents

\Chapter{Введение}
\paragraph{Тип практики:} практика по получению профессиональных умения и~опыта профессиональной деятельности.
		
\paragraph{Форма практики:}	стационарная.
\paragraph{Сроки прохождения практики:}
	c ... по~... 
\paragraph{Объем практики:} 108 часов, 1~день в~неделю.
\paragraph{Место прохождения практики:} ...

	
	
\paragraph{Цель практики:}
	получение первичных профессиональных умений и опыта профессиональной деятельности знакомство с компьютерными инструментами математического исследования

\paragraph{Задачи практики:}~\par
	\begin{enumerate}
	\item 
		Закрепление теоретических знаний, полученных в ходе обучения по направлению подготовки.
	\item 
		Отработка практических навыков, по изученным дисциплинам.
	\item 
		Получение первичных профессиональных умений и навыков.
	\item 
		Приобретение опыта самостоятельной профессиональной деятельности.
	\item 
		Адаптация студентов к исследовательской, проектной и производственной деятельности.

	\end{enumerate}
\paragraph{Тема индивидуального задания:}~\par

%TODO Указать Тему

\chapter{Производственное задание}
\section{Общая характеристика задач, решаемых в период практики}

\section{Перечень использованного программного обеспечения}

\section{Хронологический аннотированный перечень выполненных работ}


\chapter{Индивидуальное задание}


\Chapter{Заключение}
Учебная практика проходил(а) в...

В ходе учебной практики был разработан проект...

\textbf{Получены знания} ...

\textbf{Сформированы умения}...

\textbf{Освоены технологии}...
% Какие технологии освоили:
% Технологию разработки документации проекта в соответсвии с ГОСТ 19.ххх Единая система программной документации?
% Подготовка технических документов в \LaTeX?
% Проектирование образовательного ресурса?
% Разработка курса в Stepik?
% Оформление списка литературы по ГОСТ?


В рамках следующей производственной практики хотел бы решать задачи ...
% Чем хоетли бы заниматься на следующей практике.



%TODO дописать заключение	


\begin{thebibliography}{99}
%TODO дописать лиитературу	

\bibitem{Book1} \alert{Библиография оформляется по ГОСТ 7.0.5-2008} 


\end{thebibliography}

\APPENDIX
\chapter{Какое-то приложение}
\end{document}
