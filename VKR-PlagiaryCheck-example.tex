\documentclass[14pt,Diplom,PlagiaryCheck]{diplomwork}

\usepackage{fancyvrb}
\renewcommand{\theFancyVerbLine}{\footnotesize\arabic{FancyVerbLine}}

\usepackage{tikz}

\newcommand{\alert}[1]{{\color{red}#1}}

\sloppy

\date{2020}
\author{студент МКб-4301-51-00}{Фамилых Имярек Батькович}
\advisor{канд. физ.-матем. наук, доцент}{Иванов Петр Сидорович}
%\consult{cт. науч. сотрудник\\ АО ВМП <<АВИТЕК>>}{Советова София Сергеевна} % Консультант от предприятия (такое тоже бывает)
\title{Теория полуколец для младших школьников}


\napravlenie{02.03.01}{Математика и компьютерные науки}
\profile{Математические основы компьютерных наук}
\kafedra{фундаментальной математики}{Е.\,М.~Вечтомов}
\department{компьютерных и физико-математических наук}{Н.\,А.~Бушмелева}
\institute{математики и информационных систем}



\keywords{ключевые слова}

\annotation{

		Диссертация посвящена ...
		
		Цель работы: ...
	
		Центральным результатом работы является...
		
		
}

\begin{document}

	
\maketitle
\makereferat		% печатаем реферат
\newpage

\tableofcontents


\Chapter{Введение}
Выпускная квалификационная работа посвящена ...


\textbf{Актуальность} работы обусловлена...


\textbf{Объектом исследования} являются

\textbf{Предмет исследования}~--- 

\textbf{Цель работы:} ...

Для достижения поставленной цели сформулированы следующие \mbox{\textbf{задачи:}}

\begin{enumerate}
	\item ...
	\item ...
	\item ...
\end{enumerate}

\textbf{Гипотеза}, проверяемая в ходе исследования, состоит в следующем... (Гипотезу следует указывать, если ваше исследование предполагает проверку некоторой гипотезы,  сформулированной в начале Вашего исследования. Если же Вы занимаетесь разработкой программного продукта или свободным поиском свойств свойств некоторого математического объекта, то гипотеза не нужна)


Для решения сформулированных задач  применяются следующие \textbf{методы}:...

\textbf{Теоретическая значимость работы} состоит в следующем...

\textbf{Практическая значимость работы} состоит в следующем...

В целом работа носит \textbf{теоретический (или прикладной, или практический)} характер.



\textbf{Структура работы.} Выпускная квалификационная работа, общим объемом \pageref{LastPage}~стр., состоит из введения, двух глав, заключения, библиографического списка.

Первая глава посвящена ...

Вторая глава посвящена  ...


В заключении представлены основные результаты дипломной работы.

В библиографический список включено (кол-во) источников.


Результаты работы \textbf{апробированы} (в докладах..., статьях ..., внедрены...)


\chapter{что-то}

\section{sfsfsdfsfsdfsdfdsfsf}
$ f(x) = \begin{cases}
			1, &x \in I\\
			-1, &x \notin I
		 \end{cases}
$

$$
 f(x) = \begin{cases}
			1, &x \in I\\
			-1, &x \notin I
		\end{cases}
$$
\begin{equation}
\label{eq:we}
 f(x) = \begin{cases}
			 1, &x \in I\\
			 -1, &x \notin I
 \end{cases}
\end{equation}

sdfsdfsdfsdfsdf~\ref{eq:we}
sdfsdfsdf
sdf
sdfsdfsdf
fsdfsdf

\begin{theorem}[О всём сущем]
текст
\end{theorem}
\begin{proof}
текст
\end{proof}
\begin{prop}
текст
\end{prop}
\begin{proof}[Пояснение]
текст
\end{proof}
\begin{seq}
текст
\end{seq}
\begin{lemma}
текст
\end{lemma}
\begin{definition}
текст
\end{definition}
\begin{example}
текст
\end{example}
\chapter{Однако, глава}
\section{и раз}
\subsection{и два}
\subsubsection{и три}
\begin{figure}[H]
	\caption{sdfgsdfgsf}
	\label{fig:wqewe}
\end{figure}

\begin{table}[H]
	\caption{sdfgsdfgsf}
	\label{tbl:wqewe}
	\begin{tabular}{|c|c|}
		\hline
		a & b \\
		\hline
		a & b \\
		\hline
		a & b \\
		\hline
	\end{tabular}
\end{table}


\Chapter{Заключение}



\begin{thebibliography}{99}
\bibitem{Book1} \alert{Библиография оформляется по ГОСТ 7.0.5-2008} 
\end{thebibliography}

\APPENDIX
\chapter{Листинг программы}
\linespread{1}

%1.Функция,  котороя перемножает коэффициенты многочлена на матрицу:
\begin{Verbatim}[numbers=left,firstnumber=last,fontsize=\small]
f(x, F) := block([i, S], 
    S: zeromatrix(dim, dim), 
    for i: 1 thru length(F) do
        S: S+mod(F[i]*(x^^(i-1)), P), 
    return(mod(S, P))
);
\end{Verbatim}  


\end{document}
